\documentclass{article}
\usepackage{statrep}
\usepackage{parskip,xspace}
\newcommand*{\Statrep}{\mbox{\textsf{StatRep}}\xspace}
\newcommand*{\Code}[1]{\texttt{\textbf{#1}}}
\newcommand*{\cs}[1]{\texttt{\textbf{\textbackslash#1}}}
\setcounter{secnumdepth}{0}
\def\SRrootdir{/folders/myfolders}
\def\SRmacropath{/folders/myfolders/statrep_macros.sas}

\title{Statrep minimal demo}
\author{Ken Butler}
\date{}


\begin{document}

\maketitle

\section{Introduction}

This is a small document that demonstrates \texttt{statrep}. Feel free
to use this as a template for your own work.

It is designed so that you look at the source \texttt{.tex} file to
see how to do something, and, at the same time, the output
\texttt{.pdf} file to see how it comes out.

\section{The example}



First, read in some data (made-up):

\begin{Datastep}
data xx;
  input x @@;
  cards;
  10 11 14 15 16 18 23 32 57
  ;
\end{Datastep}

Then, obtain a listing of the data:

\begin{Sascode}[store=a]
proc print;  
\end{Sascode}

and print it out:

\Listing[store=a]{aa}

Next, make a boxplot:

\begin{Sascode}[store=b]
proc sgplot;
  vbox x;
\end{Sascode}

and display it, noting that it is a graph this time:

\Graphic[store=b]{bb}

We have a serious upper outlier, and even beyond that, some
right-skewedness, so that the mean is a lot bigger than the median.

Finally, note that the output from \texttt{proc univariate} is rather
long (displayed tiny below). I am going to test that the ``centre''
(mean or median) is 10.5, which the boxplot suggests that it isn't:

\begin{Sascode}[store=c]
proc univariate mu0=10.5;
  var x;
\end{Sascode}

\Listing[store=c,fontsize=tiny]{cc}

Suppose we only want the ``tests for location'', and we don't care
about the rest of it. This is one of the ``objects'' in the output. To
find out what it is called, go back to SAS Studio and look at the Log
tab (or look at the \texttt{.log} file if you are running SAS from the
command line). It's at the bottom of the log output; it's called
\texttt{Univariate.x.TestsForLocation}. The first thing says what
\texttt{proc} it is, the second says what variable it is (\texttt{x}),
and the third says which table it is. You only need the third bit, and
it is \emph{not} case sensitive.\footnote{So you do not need to
  preserve the CamelCase.}
The \texttt{store=} must be the \emph{same} and the final label
in the curly brackets must be \emph{different} from before:

\Listing[store=c,objects=testsforlocation]{cd}

At $\alpha=0.05$, the $t$-test and the sign test disagree. Probably
what is happening is that high outlier is inflating the sample SD
(grossly: the SD is 15), so that the sample mean (22) is not so much
higher than 
the hypothesized mean (10.5) \emph{relative to the SD}. The $t$-statistic is
2.28, which, with a small sample size, is not quite big enough to
reach a P-value as small as 0.05.

The sign test is telling a different story. In the data, there are 8
values bigger than 10.5 and only one below. Even with a small sample
size of 9, this is the kind of imbalance above and below that you would
be unlikely to see by chance.

The signed-rank test doesn't apply here (that requires approximately
symmetric data). So I would trust the sign-test result and not the others.

\end{document}
