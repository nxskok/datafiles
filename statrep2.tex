\documentclass{article}\usepackage[]{graphicx}\usepackage[]{color}
%% maxwidth is the original width if it is less than linewidth
%% otherwise use linewidth (to make sure the graphics do not exceed the margin)
\makeatletter
\def\maxwidth{ %
  \ifdim\Gin@nat@width>\linewidth
    \linewidth
  \else
    \Gin@nat@width
  \fi
}
\makeatother

\definecolor{fgcolor}{rgb}{0.345, 0.345, 0.345}
\newcommand{\hlnum}[1]{\textcolor[rgb]{0.686,0.059,0.569}{#1}}%
\newcommand{\hlstr}[1]{\textcolor[rgb]{0.192,0.494,0.8}{#1}}%
\newcommand{\hlcom}[1]{\textcolor[rgb]{0.678,0.584,0.686}{\textit{#1}}}%
\newcommand{\hlopt}[1]{\textcolor[rgb]{0,0,0}{#1}}%
\newcommand{\hlstd}[1]{\textcolor[rgb]{0.345,0.345,0.345}{#1}}%
\newcommand{\hlkwa}[1]{\textcolor[rgb]{0.161,0.373,0.58}{\textbf{#1}}}%
\newcommand{\hlkwb}[1]{\textcolor[rgb]{0.69,0.353,0.396}{#1}}%
\newcommand{\hlkwc}[1]{\textcolor[rgb]{0.333,0.667,0.333}{#1}}%
\newcommand{\hlkwd}[1]{\textcolor[rgb]{0.737,0.353,0.396}{\textbf{#1}}}%

\usepackage{framed}
\makeatletter
\newenvironment{kframe}{%
 \def\at@end@of@kframe{}%
 \ifinner\ifhmode%
  \def\at@end@of@kframe{\end{minipage}}%
  \begin{minipage}{\columnwidth}%
 \fi\fi%
 \def\FrameCommand##1{\hskip\@totalleftmargin \hskip-\fboxsep
 \colorbox{shadecolor}{##1}\hskip-\fboxsep
     % There is no \\@totalrightmargin, so:
     \hskip-\linewidth \hskip-\@totalleftmargin \hskip\columnwidth}%
 \MakeFramed {\advance\hsize-\width
   \@totalleftmargin\z@ \linewidth\hsize
   \@setminipage}}%
 {\par\unskip\endMakeFramed%
 \at@end@of@kframe}
\makeatother

\definecolor{shadecolor}{rgb}{.97, .97, .97}
\definecolor{messagecolor}{rgb}{0, 0, 0}
\definecolor{warningcolor}{rgb}{1, 0, 1}
\definecolor{errorcolor}{rgb}{1, 0, 0}
\newenvironment{knitrout}{}{} % an empty environment to be redefined in TeX

\usepackage{alltt}
\usepackage{statrep}
\usepackage{parskip,xspace}
\newcommand*{\Statrep}{\mbox{\textsf{StatRep}}\xspace}
\newcommand*{\Code}[1]{\texttt{\textbf{#1}}}
\newcommand*{\cs}[1]{\texttt{\textbf{\textbackslash#1}}}
\setcounter{secnumdepth}{0}
\def\SRrootdir{/folders/myfolders}
\def\SRmacropath{/folders/myfolders/statrep_macros.sas}

\title{Statrep-Sweave minimal demo}
\author{Ken Butler}
\date{}
\IfFileExists{upquote.sty}{\usepackage{upquote}}{}
\begin{document}

\maketitle

\section{Introduction}

This is a small document that demonstrates \texttt{statrep} for SAS
and Sweave for R, combined. Feel free
to use this as a template for your own work.

It is designed so that you look at the source \texttt{.tex} file to
see how to do something, and, at the same time, the output
\texttt{.pdf} file to see how it comes out.

\section{The example}

I have intermingled R and SAS code below, to show that it can be
arranged in any order.

First, read in some data (made-up) into SAS:

\begin{Datastep}
data xx;
  input x @@;
  cards;
  10 11 14 15 16 18 23 32 57
  ;
\end{Datastep}

and into R:

\begin{knitrout}
\definecolor{shadecolor}{rgb}{0.969, 0.969, 0.969}\color{fgcolor}\begin{kframe}
\begin{alltt}
\hlstd{x}\hlkwb{=}\hlkwd{c}\hlstd{(}\hlnum{10}\hlstd{,}\hlnum{11}\hlstd{,}\hlnum{14}\hlstd{,}\hlnum{15}\hlstd{,}\hlnum{16}\hlstd{,}\hlnum{18}\hlstd{,}\hlnum{23}\hlstd{,}\hlnum{32}\hlstd{,}\hlnum{57}\hlstd{)}
\end{alltt}
\end{kframe}
\end{knitrout}

Then, obtain a listing of the data:

\begin{Sascode}[store=a]
proc print;  
\end{Sascode}

and print it out:

\Listing[store=a]{aa}

This is, of course, much simpler in R:

\begin{knitrout}
\definecolor{shadecolor}{rgb}{0.969, 0.969, 0.969}\color{fgcolor}\begin{kframe}
\begin{alltt}
\hlstd{x}
\end{alltt}
\begin{verbatim}
## [1] 10 11 14 15 16 18 23 32 57
\end{verbatim}
\end{kframe}
\end{knitrout}

Next, make a boxplot:

\begin{Sascode}[store=b]
proc sgplot;
  vbox x;
\end{Sascode}

and display it, noting that it is a graph this time:

\Graphic[store=b]{bb}

and in R:

\begin{knitrout}
\definecolor{shadecolor}{rgb}{0.969, 0.969, 0.969}\color{fgcolor}\begin{kframe}
\begin{alltt}
\hlkwd{boxplot}\hlstd{(x)}
\end{alltt}
\end{kframe}
\includegraphics[width=\maxwidth]{figure/unnamed-chunk-3-1} 

\end{knitrout}

The R boxplot does not show the mean, and is thus more in line with
the original Tukey conception of the boxplot.

\end{document}
